% arara: xelatex: {synctex: true}
% arara: indent: {overwrite: yes}
\documentclass[]{IMTexam}

\usepackage[enums]{IMTtikz}

\givecredits
\author{Isabella B.}
\USPN{11810773}
\date{}
\lecture{Física I} % disciplina
\lcode{4302111}
\hwtype{Resolução} % o que é
\examname{Provinha VIII} % prova

\begin{document}

\maketitle

\begin{questions}

	\question \label{ques:q1} Em 19 de janeiro de 2006, a NASA lançou a sonda \textit{New Horizons }numa missão para estudar a geomorfologia de Plutão. No entanto, ela não foi lançada diretamente para o planeta-anão, mas numa trajetória de \SI{23}{\kilo\meter\per\second} (em relação ao Sol) em direção a Júpiter, astro com que teve máxima aproximação no dia 28 de fevereiro de 2007, e cuja força de gravidade foi usada para acelerá-la, um processo conhecido como assistência gravitacional (ou estilingue gravitacional).

	\begin{parts}
		\part \label{part:q1a} Mostre que, considerando situações inicial e final em que a sonda está suficientemente distante de Júpiter, o processo de deflexão de sua trajetória pode ser considerado uma colisão elástica.
		
		\begin{solution}
			Considerando que a sonda viaja à uma distância $ R $ do centro de massa (CM) de Júpiter, teremos a conservação de energia e de momento do sistema se, e somente se, as únicas forças atuantes forem conservativas, o que é o caso para um $ R $ suficientemente grande, de tal forma que a sonda orbita o planeta numa região sem atmosfera e, portanto, sem forças dissipativas.
			
			Como as forças que atuam no sistema são todas internas, pois a sonda e o planeta sofrem somente uma força gravitacional mútua, o momento é conservado\footnote{$\sum\vec{F_{int}}$ representa as forças internas ao sistema, enquanto $ \vec{F_{ext}} $ representa as forças externas, e $\sum\vec{p}$ é o momento total.}:
			\[ \sum \vec{F_{int}}=\vec{0},\qzuad \vec{F_{ext}}=\vec{0}\implies \sum \dod{\vec{p}}{t}=\vec{0} \]
			
			Dessa forma, a energia do sistema também é conservada e, portanto, podemos considerar que o problema trata de uma colisão elástica.
		\end{solution}

		\part \label{part:q1b} O momento angular de um corpo na posição $\vec{r}$ com momento $\vec{p}$ é dado pela expressão $ \vec{L}=\vec{r}\times\vec{p} $. Calcule a variação temporal de $\vec{L}$ para um corpo sob ação da força gravitacional de um planeta (muito massivo em relação ao objeto) na origem.

		\begin{solution}
			Denotarei o módulo de um vetor $\vec{\alpha}$ por $ |\vec{\alpha}|=\alpha $.
			
			Para um corpo de massa $ m $ e posição $ \vec{r} $ em relação a um corpo de massa $ M\gg m $ que exerce sobre ele uma força gravitacional, temos $ \ddot{r}=g_p $, que é a gravidade do planeta.
			
			Nessa situação, podemos descrever seu vetor posição como
			\[ \vec{r}=r\del{\cos\theta\,\ihat+\sin\theta\,\jhat} \]
			onde $ \theta $ é o ângulo formado em relação a alguma reta bem definida que passa pelo centro de massa do planeta.
			
			Dessa forma
			\begin{align}
			\ddot{\vec{r}}&=\dod[2]{}{t}\sbr[1]{r\del[1]{\cos\theta\,\ihat+\sin\theta\,\jhat}}\nonumber\\
			&=\dod{}{t}\sbr{\dot{r}\del{\cos\theta\,\ihat+\sin\theta\,\jhat}+r\del{-\dot{\theta}\sin\theta\,\ihat+\dot{\theta}\cos\theta\,\jhat}}\nonumber\\
			\intertext{se a única força que atua no sistema é a gravitacional, $ \dot{\theta}=0 $}
			&=\dod{}{t}\sbr[1]{\dot{r}\del{\cos\theta\,\ihat+\sin\theta\,\jhat}}\nonumber\\
			&=\ddot{r}\del{\cos\theta\,\ihat+\sin\theta\,\jhat}
			+\cancelto{0}{\dot{r}\,\dot{\theta}\del{-\sin\theta\,\ihat+\cos\theta\,\jhat}}\nonumber\\
			\ddot{\vec{r}}&=g_p\del{\cos\theta\,\ihat+\sin\theta\,\jhat}\label{eq:ddotr}
			\end{align}
			
			Sendo $ \vec{p}=m\,\vec{v}=m\,\dot{\vec{r}} $, e a derivada do produto vetorial
			\begin{align*}
			\dod{\vec{L}}{t}&=\dod{}{t}\sbr{\vec{r}\times\vec{p}}=\dod{\vec{r}}{t}\times\vec{p}+\vec{r}\times\dod{\vec{p}}{t}\\
			\intertext{substituindo $ \vec{p}=m\,\dot{\vec{r}} $, temos}
			&=\cancelto{0}{\dod{\vec{r}}{t}\times m\,\dot{\vec{r}}}+\vec{r}\times\dod{m\,\dot{\vec{r}}}{t}\\
			&=m\,\vec{r}\times\ddot{\vec{r}}\\
			\intertext{substituindo \ref{eq:ddotr} e tomando o produto vetorial, temos}
			&=m\,\begin{vmatrix}
			\ihat & \jhat & \khat\\
			r\,\cos\theta & r\,\sin\theta & 0\\
			g_p\,\cos\theta & g_p\,\sin\theta & 0
			\end{vmatrix}\\
			\dod{\vec{L}}{t}&=m\,\khat\del{r\,g_p\del{\cos\theta\,\sin\theta-\cos\theta\,\sin\theta}}=\vec{0}
			\end{align*}
		\end{solution}

		Pelo item anterior, sabemos que o movimento ocorre num plano. Assim, estabeleçamos espertamente um referencial $ Oxy $ com origem inicialmente em Júpiter\footnote{A interação entre a sonda e Júpiter dura alguns dias, bastante pouco face à translação 	joviana de quase 12 anos. Assim, podemos considerar que, ao longo do processo de colisão, o referencial não estará acelerado pela gravidade do Sol.}, e cujo eixo $ Ox $ seja paralelo a seu momento final.

		\part \label{part:q1c} Calcule, nesse referencial, a velocidade final da sonda em termos das velocidades iniciais da sonda e de Júpiter. Considere que a massa da \textit{New Horizons }é desprezível em relação à de um planeta.
		
		\begin{solution}
			
			Sejam $ m $ a massa da sonda, $ M $ a massa do planeta, e $ \vec{v},\vec{p} $ e $ \vec{V},\vec{P} $ suas respectivas velocidades e momentos em relação ao Sol.
			
%			\begin{multi}
				Posicionamos o eixo de coordenadas de tal forma que $ Ox $ seja paralela ao momento final de Júpiter $\vec{P_f}$. Seu momento inicial $ \vec{P_i} $, e os momentos inicial e final da sonda ($\vec{p_i}$ e $ \vec{p_f} $) fazem ângulos $ \theta $, $ \alpha $ e $ \beta $, respectivamente, com $ Ox $.
				
%				\paragraph{Nota:} Repare que, por definição, $ Oy $ é perpendicular à $ Ox $ e $ \vec{P} $.
				
%				\nach
%				\begin{center}
%					\begin{tikzpicture}
%					\fill (0,0) coordinate (O) circle (2pt) node[below right] {$ O $};
%					\draw[->] (-2,0) -- (2,0) coordinate (x) node[right] {$ x $};
%					\draw[->] (0,-2) -- (0,2) node[above] {$ y $};
%					\draw[dashed] (0,0) circle (1);
%					
%					\draw[-Latex] (0,1) -- node[above] {$ \vec{P} $} +(1,0);
%					\draw[Latex-] (-30:-1) coordinate (a) -- node[above left] {$ \vec{p_i} $} +(60:-1);
%					\path[name path=O--ai] (O) -- +(-2,0);
%					\path[name path=a--ai] (a) -- +(60:-3);
%					\path[name intersections={of=O--ai and a--ai,by=ai}];
%					\draw[Latex-] (0,0) -- (30:-2) coordinate (V) node[right] {$ \vec{P_i} $};
%%					\draw[dotted] (a) -- (ai);
%					
%					\pic[draw,angle radius=10pt,angle eccentricity=1,"$ \mathsmaller{\theta} $" {xshift=4pt,yshift=2pt,fill=white,inner sep=1pt}] {angle=O--ai--a};
%					
%					\pic[draw,angle radius=10pt,angle eccentricity=1,"$ \alpha $" {xshift=5pt,yshift=1pt}] {angle=x--O--V};
%					\end{tikzpicture}
%				\end{center}
%			\end{multi}
			
			Por transformações de Galileu, sabemos que a velocidade inicial da sonda no referencial do planeta será $ \vec{v_i'}=\vec{v_i}-\vec{V_i} $.
			
			Considerando uma colisão elástica entre os corpos, como no item \ref{part:q1a}, por conservação de momento, temos:
			\[ \vec{p_i}+\vec{P_i}=\vec{p_f}+\vec{P_f}\implies \vec{p_f}-\vec{p_i}=-M\,\del{\vec{V_f}-\vec{V_i}} \]
			E, como $ M\gg m $ e o encontro é muito rápido, a sonda afeta a velocidade do planeta desprezivelmente, portanto, podemos considerar $ \vec{p_f}'\approx \vec{p_i}' $ no referencial do planeta. Dessa forma
			\begin{align*}
				\vec{v_f}'&=\vec{v_i}'\\
				\vec{v_f}'&=\vec{v_i}-\vec{V_i}
			\end{align*}
			
			\begin{multi}
				Porém, como a sonda muda de direção no encontro, essas velocidades não são iguais em módulo, dessa forma
				\begin{equation}\label{eq:vfNH}
				|\vec{v_f}'|=v_i^{2}+V_i^{2}-2v_i\,V_i\cos \del{\theta-\alpha}
				\end{equation}
				
				\nach
				\centering
				\begin{tikzpicture}[scale=2]
					\draw[-Latex] (0,0) coordinate (O) -- node[below] {$ \vec{v_i} $} (1,0) coordinate (a);
					\draw[-Latex] (a) -- node[below right] {$ -\vec{V_i} $} +(30:2) coordinate (b);
					\draw[-Latex] (O) -- node[above left] {$ \vec{v_f}' $} (b);
					
					\draw[dashed] (a) -- +(1,0) coordinate (c);
					
					\pic[draw,angle radius=15pt,angle eccentricity=1,"$ \mathsmaller{\pi+\theta-\alpha} $" {xshift=18pt,yshift=1pt,fill=white,inner sep=1pt}] {angle=c--a--b};
				\end{tikzpicture}
			\end{multi}
			
			
			
%			Assumindo as condições dadas no item \ref{part:q1a}, consideramos a interação entre a sonda e o planeta um colisão elástica. Desse modo, temos, no referencial do planeta, a conservação do momento e da energia em $ x $ e em $ y $.
%			
%			Sendo $ p_x',p_y' $ as componentes $ x $ e $ y $ do momento da sonda no referencial do Sol, temos:
%			\begin{gather}
%				p_{x\,i}+P_{x\,i}=p_{x\,f}+P_{x\,f}\label{eq:px1}\\
%				p_{y\,i}+P_{y\,i}=p_{y\,f}+P_{y\,f}\label{eq:py1}\\
%				\dfrac{1}{2}m\,v_{x\,i}^{2}+\dfrac{1}{2}M\,V_{x\,i}^{2}=\dfrac{1}{2}m\,v_{x\,f}^{2}+\dfrac{1}{2}M\,V_{x\,f}^{2}\label{eq:ex1}\\
%				\dfrac{p_{y\,i}^{2}}{2m}+\dfrac{P_{y\,i}^{2}}{2M}=\dfrac{p_{y\,f}^{2}}{2m}+\dfrac{P_{x\,f}^{2}}{2M}\label{eq:ey1}
%			\end{gather}
			
%			Como praticamente não há mudança de direção no momento do planeta, $ p_{y\,i}=p_{y\,f}+P_{y\,f}-P_{y\,i}\approx p_{y\,f} $, e podemos considerar o caso de colisão unidimensional.
			
%			Pela derivação do H. M. Nussenzveig, vol. 1, pp. 171-173, temos
%			\begin{equation}\label{eq:HMs}
%				v_{x\,f}=\dfrac{2M}{M+m}V_{x\,i}-\dfrac{M-m}{M+m}v_{x\,i}
%			\end{equation}
%			
%			No limite de $ M\gg m $, temos
%			\[ v_{x\,f}\approx 2V_{x\,i}-v_{x\,i} \]
%			
%			E, no referencial do planeta, temos
%			\[ v_{x\,f}\approx 2V_{x\,i}-v_{x\,i} \]
%			Pela discussão do livro Curso de Física Básica, vol. 1 -- H. M. Nussenzveig, pp. 171-173, temos que a velocidade relativa final vai ser oposta a velocidade relativa inicial, portanto
%			\begin{equation}\label{eq:vxfp}
%				v_{x\,f}'=-v_{x\,i}'\implies v_{x\,f}'=V_{x\,i}-v_{x\,i}
%			\end{equation}
			
		\end{solution}

		\part \label{part:q1d} Júpiter percorre sua órbita a uma velocidade média de \SI{13}{\kilo\meter\per\second}. Comente qual seria a situação mais favorável para a assistência gravitacional (e.g. em termos das velocidades dos corpos envolvidos) e diga, nesse caso hipotético, qual seria a variação da velocidade da \textit{New Horizons} no referencial do Sol.
		
		\begin{solution}
			Como a velocidade final no referencial do Sol será
			\begin{align*}
				\envert{\vec{v_f}-\vec{V_f}}&\approx \envert{\vec{v_f}-\vec{V_i}}=v_i^{2}+V_i^{2}-2v_i\,V_i\cos \del{\theta-\alpha}\\
				v_f^{2}+V_i^{2}-2v_f\,V_i\cos\del{\beta-\theta}&=v_i^{2}+V_i^{2}-2v_i\,V_i\cos \del{\theta-\alpha}\\
				\del{v_f-\del{V_i\cos\del{\beta-\theta}}}^{2}&=v_i^{2}-2v_i\,V_i\cos \del{\theta-\alpha}+\del{V_i\cos\del{\beta-\theta}}^{2}\\
				v_f&=\sqrt{v_i^{2}-2v_i\,V_i\cos \del{\theta-\alpha}+\del{V_i\cos\del{\beta-\theta}}^{2}}+V_i\cos\del{\beta-\theta}
			\end{align*}
			Podemos ver que o valor máximo da raiz, em relação aos ângulos, se dá quando $ \theta-\alpha=\pi $ e $ \beta-\theta=0 $, o que corresponde à sonda entrar em contato com o planeta com velocidade contrária a sua, e sair no mesmo sentido e direção de viagem do planeta. Desse modo, sua variação de velocidade seria:
			\begin{align}
				\Delta v&=v_{f\,max}-v_i\nonumber\\
				&=\sqrt{v_i^{2}+2v_i\,V_i+V_i^{2}}+V_i-v_i\nonumber\\
				\Delta v&=v_i+V_i+V_i-v_i=2V_i\label{eq:deltaV}\\
				&=2\cdot13=\SI{26}{\kilo\meter\per\second}\nonumber
			\end{align}
%			 $ \Delta v=v_{x\,f}-v_{x\,i}\approx 2\del{V_{x\,i}-v_{x\,i}}\approx 26-2v_{x\,i} $.
		\end{solution}

		\part \label{part:q1e} Sabendo que a \textit{New Horizons }tem massa de \SI{480}{\kilo\gram} e tem velocidade de \SI{4}{\kilo\meter\per\second} antes de utilizar a assistência gravitacional, estime quanto combustível seria necessário para que apenas a sonda\footnote{Como trata-se de uma estimativa, estamos desconsiderando que a \textit{New Horizons} precisaria levar consigo a massa de combustível. Portanto a quantidade estimada de combustível é subestimada.} ganhasse a mesma velocidade adquirida via assistência gravitacional no caso mais favorável. Considere que seria utilizado como combustível LOX/LH2 com 100\% de eficiência (o que claramente não é possível).
		
		\paragraph{Dados:} A reação de combustão é dada por 2H$_2\,+\,$O$_2 \to $2H$_2$O$\,+\,$energia, sendo que o calor específico de combustão da reação é de \SI{141,8}{\mega\joule\per\kilogram}.
		
		\begin{solution}
			Pelo item anterior, a sonda ganharia $ \SI{26}{\kilo\meter\per\second} $.
			
			Considerando a variação de energia cinética da sonda onde ela ganharia essa velocidade, temos
			\begin{align*}
				\Delta K&=\dfrac{1}{2}m\del{v_f^{2}-v_i^{2}}\\
				&=\dfrac{1}{2}480\del{(26+4)^{2}-4^{2}}\\
				&=240(26+4-4)(26+4+4)=\SI{212160}{\mega\joule} 
			\end{align*}
			portanto, precisaríamos de, aproximadamente
			\[ \dfrac{\num{212160}}{\num{141.8}}=\SI{1496.2}{\kilogram} \]
			de combustível.
		\end{solution}
	\end{parts}

\end{questions}
\end{document}
