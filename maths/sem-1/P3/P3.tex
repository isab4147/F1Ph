\documentclass{IMTexam}

\usepackage[enums]{IMTtikz}

\givecredits
\author{Isabella B. Amaral}
\USPN{118010773}
\lecture{Matemática I}
\examname{Prova III}
\hwtype{Resolução}
\lcode{}
\date{30 de novembro}

\newtheorem{theorem}{Teorema}[question]
\newtheorem{corollary}{Corolário}[theorem]
\newtheorem{lemma}[theorem]{Lema}
\let\oldemptyset\emptyset
\let\emptyset\varnothing
\newcommand\restrict[1]{{% we make the whole thing an ordinary symbol
		\left.\kern-\nulldelimiterspace % automatically resize the bar with \right
		% the function
		\vphantom{\big|} % pretend it's a little taller at normal size
		\right|_{#1} % this is the delimiter
}}

\begin{document}
	
	\maketitle
	
	\paragraph{Nota:} Todos os teoremas e axiomas referenciados nessa prova são citados pelo Apostol nos capítulos abordados em aula, a não ser que seja explicitado o contrário.
	
	\begin{questions}
		\titledquestion{Grupo I --- Q1}
		
		Calcule os seguintes limites:
		
		\begin{parts}
			\part $ \lim\limits_{x\to0}x^{p}\cos\dfrac{1}{x^{q}}, p $ e $ q $ naturais.
			
			\begin{solution}
				Seja $ \alpha = 1/x^{q} $, $\alpha\to\infty$ quando $ x\to0 $, porém sendo a função cosseno limitada, pelo teorema 3.3, nós temos que
				\begin{align*}
					-1\leqslant&\cos\alpha\leqslant1\\
					-x^{p}\leqslant&x^{p}\cos\alpha\leqslant x^{p}\\
					\intertext{e quando $ x\to0 $, temos}
					0\leqslant&\lim\limits_{x\to0}x^{p}\cos\alpha\leqslant 0
				\end{align*}
			o que nos dá \[ \lim\limits_{x\to0}x^{p}\cos\dfrac{1}{x^{q}}=0. \]
			\end{solution}
		
			\part $\lim\limits_{t\to1}\dfrac{x^{3}-1}{x-1}$.
			
			\begin{solution}
				Assumindo um erro de digitação no limite, temos, pela fatoração $ a^{3}-b^{3}=\del{a-b}\del{a^{2}+a\,b+b^{2}} $ que
				\begin{align*}
					\lim\limits_{x\to1}\dfrac{x^{3}-1}{x-1}&=\lim\limits_{x\to1}\dfrac{\del{x-1}\del{x^{2}+x+1}}{x-1}\\
					&=\lim\limits_{x\to1}x^{2}+x+1,
				\end{align*}
			que pode ser calculado substituindo $ x $ por $ 1 $, simplesmente, e nos dá
			\[ 1^{2}+1+1=3. \]
			\end{solution}
			
			\part $ \lim\limits_{t\to0}\dfrac{\sqrt{2+t}-\sqrt{2-t}}{t}. $
			
			\begin{solution}
				Manipulando a expressão, temos
				\begin{align*}
					\lim\limits_{t\to0}\dfrac{\sqrt{2+t}-\sqrt{2-t}}{t}&=\lim\limits_{t\to0}\dfrac{\del{\sqrt{2+t}-\sqrt{2-t}}\del{\sqrt{2+t}+\sqrt{2-t}}}{t\del{\sqrt{2+t}+\sqrt{2-t}}}\\
					&=\lim\limits_{t\to0}\dfrac{2+t-\del{2-t}}{t\del{\sqrt{2+t}+\sqrt{2-t}}}\\
					&=\lim\limits_{t\to0}\dfrac{2}{\sqrt{2+t}+\sqrt{2-t}}\\
					\intertext{novamente, substituindo $ t $ por $ 0 $, temos}
					&=\dfrac{2}{2\sqrt{2}}=\dfrac{\sqrt{2}}{2}.
				\end{align*}
			\end{solution}
			
			\part $ \lim\limits_{x\to0}\dfrac{\sin\del{\tan x}}{\sin x} $.
			\begin{solution}
				Manipulando a expressão temos que
				\begin{align*}
					\lim\limits_{x\to0}\dfrac{\sin\del{\tan x}}{\sin x}\cdot\dfrac{\tan x}{\tan x}&=\lim\limits_{x\to0}\dfrac{\sin\del{\tan x}}{\tan x}\cdot\dfrac{1}{\cos x}\\
					\intertext{e pelo teorema 3.1, temos que}
					\lim\limits_{x\to0}\dfrac{\sin\del{\tan x}}{\tan x}\cdot\dfrac{1}{\cos x}&=\del{\lim\limits_{x\to0}\dfrac{1}{\cos x}}\cdot\del{\lim\limits_{x\to0}\dfrac{\sin\del{\tan x}}{\tan x}}.
				\end{align*}
				Sendo $ \lim\limits_{x\to0}\tan x=0 $ temos o limite fundamental $ \lim\limits_{x\to0}\sin\del{\tan x}{\tan x}=1 $, e como $ x\to0 \implies \cos x\to 1 $, o limite desejado é
				\[ \del{\lim\limits_{x\to0}\dfrac{1}{\cos x}}\cdot\del{\lim\limits_{x\to0}\dfrac{\sin\del{\tan x}}{\tan x}}=1\cdot1=1. \]
			\end{solution}
		\end{parts}
	
		\titledquestion{Grupo II --- Q2}
		
		Sejam $ p $ e $ q $ naturais, calcule $ \lim\limits_{x\to0}\dfrac{\sin px-\sin qx}{x} $.
		
		\begin{solution}
			Primeiramente, temos um lema
			
			\begin{lemma}
				Seja $ \lim\limits_{x\to0}\dfrac{\sin x}{x}=1 $ o limite fundamental. Aplicando um fator linear $ \alpha\neq 0 $ qualquer no seno, temos
				$ \lim\limits_{x\to0}\dfrac{\sin \alpha x}{x}=\alpha $.
			\end{lemma}
		
			\begin{proof}
				Basta manipular o limite de tal forma que
				\[ \lim\limits_{x\to0}\dfrac{\sin \alpha x}{x}\cdot\dfrac{\alpha}{\alpha}, \]
				e pelo teorema 3.1, podemos remover a constante do limite de tal forma que
				\[ \alpha\cdot\del{\lim\limits_{x\to0}\dfrac{\sin \alpha x}{\alpha x}}. \]
				Como o produto $ \alpha x $ tende a zero da mesma forma que $ x $ sozinho, o limite fundamental permanece inalterado e temos
				\[ \lim\limits_{x\to0}\dfrac{\sin \alpha x}{x}=\alpha\cdot1=\alpha. \]
%				como se queria demonstrar.
			\end{proof}
			
			Pelo teorema 3.1, temos que
			\[ \lim\limits_{x\to0}\dfrac{\sin px-\sin qx}{x}=\lim\limits_{x\to0}\dfrac{\sin px}{x}-\lim\limits_{x\to0}\dfrac{\sin qx}{x} \]
			que é, trivialmente, a subtração de dois limites como o discutido no lema acima, portanto, temos
			\[ \lim\limits_{x\to0}\dfrac{\sin px}{x}-\lim\limits_{x\to0}\dfrac{\sin qx}{x}=p-q. \]
			
			Note que a restrição no lema de $ \alpha\neq0 $ também se aplica ao limite desejado e, portanto, caso $ p=0 $ ou $ q=0 $ teremos uma indeterminação.
		\end{solution}
	
		\titledquestion{Grupo II --- Q3}
		
		Determine todas as funções $ f\colon \mathbb{R}\longleftrightarrow\mathbb{R} $ contínuas tais que existe $ \lim\limits_{x\to0}f(x)\sin\dfrac{1}{x}. $
		
		\begin{solution}
			Seja $ \alpha = 1/x^{q} $, $\alpha\to\infty$ quando $ x\to0 $, porém sendo a função seno limitada, pelo teorema 3.3, nós temos que
			\begin{align*}
				-1\leqslant&\sin\alpha\leqslant1\\
				-f(x)\leqslant&f(x)\sin\alpha\leqslant f(x).
			\end{align*}
			Dessa forma, podemos notar que o limite desejado só existe se $ \lim\limits_{x\to0}-f(x)=\lim\limits_{x\to0}f(x) $, e isso só deve ocorrer para o caso em que $ \lim\limits_{x\to0}f(x)=0 $, pois caso esta tenda à $ \infty $ ou $ -\infty $ teremos uma descontinuidade em $ x=0 $.
		\end{solution}
		
		\titledquestion{Grupo III --- Q4}
		
		Seja $ f : \mathbb{R} \to \mathbb{R} $ contı́nua e periódica de perı́odo $ p > 0 $. Prove que $ f $ é uniformemente contı́nua.
		
		\begin{solution}
			Sendo a função $ f $ contínua, para um intervalo $ \intcc{0,p} $ esta deve ser uniformemente contínua pois, dados $ x,y $ quaisquer no intervalo, pela continuidade da função sempre haverá $ \delta_i $ tal que, dado um $ \varepsilon_i>0, $
			\[ \envert{y-x}<\delta_i\implies\envert{f(y)-f(x)}<\varepsilon_i, \]
			pois, pelo teorema 3.13, podemos subdividir os valores da função em partições menores que $ \varepsilon $, e isso nos permite escolher o menor $ \delta_i(\varepsilon_i) $, o qual satisfará todas as inequações simultaneamente\footnote{Alternativamente, basta assumir um intervalo fechado e compacto, para o qual, então, será possível encontrar finitos $ \delta_i(\varepsilon_i) $ e, então, escolher o melhor/menor dentre eles para garantir a uniformidade da função no intervalo.}.
			
			Dessa forma, sendo a função periódica, devemos ter que é uniformemente contínua para qualquer intervalo $ \intcc{k\,p,(k+1)p},k\in\mathbb{Z} $, pois, dados $ x_k=x+k\,p,y_k=y+k\,p, $ temos
			\[ \envert{y_{k}-x_k}=\envert{y+k\,p-\del{x+k\,p}}=\envert{y-x}<\delta\implies\envert{f(y)-f(x)}<\varepsilon. \]
			
			Dada a validade da afirmação para qualquer $ k\in\mathbb{Z} $, a função $ f $ é uniformemente contínua em todo o seu domínio ($ k $ vezes um intervalo fechado de interesse gera todo o domínio real) como se queria demonstrar.
		\end{solution}
		
		\titledquestion{Grupo IV --- Q7}
		
		Seja $ f : \mathbb{R} \to \mathbb{R} $ definida por $ f(x) = \cos(3x) $,
		se $ x \leqslant \alpha $, e $ f(x) = x − \alpha $, se $ x > \alpha $. Determine os $ \alpha\in\mathbb{R} $ para os quais $ f $ fica contı́nua em todos os pontos de $ \mathbb{R} $.
		
		\begin{solution}
			Como $ \cos(3x) $ é contínua e $ x-\alpha $ também, $ f(x) $ será contínua se, para $ x=\alpha $, tivemos $ f(\alpha)=\cos(3\alpha)=\alpha-\alpha=0 $.
			
			Dessa forma, $ f(x) $ será contínua para $ \alpha=\dfrac{1}{3}\arccos\del{\cos0}+2n\,\pi\implies \alpha=\dfrac{\pi}{6}+2n\,\pi$ ou $\alpha=\dfrac{\pi}{2}+2n\,\pi,n\in\mathbb{Z} $.
		\end{solution}
		
		\titledquestion{Grupo V --- Q11}
		
		Seja $ f : \intcc{a,b} \to \mathbb{R} $ convexa.
		\begin{enumerate}[label=\roman*.]
			\item Prove que $ f $ tem máximo e esse máximo é $ f(a) $ ou $ f(b) $.
			\item Prove que $ f $ é limitada inferiormente.
		\end{enumerate}
	
		\begin{solution}
			Primeiro assumimos a continuidade de funções convexas.
			
			Sendo a função convexa, podemos dividí-la em duas porções monotônicas, as quais chamarei de $ \ell $ e $ r $, respectivamente. $ \ell\colon\intcc{a,c}\longrightarrow\mathbb{R} $ a parte monotonicamente decrescente e $ r\colon\intcc{c,b}\longrightarrow\mathbb{R} $ a parte monotonicamente crescente.
			
			Dessa forma, temos que para dois pontos subsequentes $ x<y $ no domínio de $ \ell $, $ \ell(x)>\ell(y) $ e, portanto, $ \ell(a)>\ell(c) $ e, de forma análoga, $ r(c)< r(b) $. 
			
			Portanto (i), $ \max f $ no intervalo $ \intcc{a,b} $ deve ser $ f(a)=\ell(a) $ ou $ f(b)=r(b) $, e (ii) a função é limitada inferiormente por $ L\leqslant f(c)=\ell(c)=r(c), L\in\mathbb{R} $ (o que também pode ser garantido pelo teorema 3.12).
		\end{solution}
		
		\titledquestion{Grupo VI --- Q12}
		
		Sejam $ a, b, c, h $ em $ \mathbb{R} $, com $ h > 0 $, e $ f(t) = a\,t^{2} + b\,t + c \geqslant 0 $, para $ 0 \leqslant t \leqslant h $. Considere um sólido convexo $ S $ contido na região $ \set{(x, y, z) \in \mathbb{R}^{3} : 0 \leqslant x \leqslant h} $. Suponha que a seção de $ S $ perpendicular ao eixo dos $ x $ que passa por $ (\xi, 0, 0) $, para $ 0 \leqslant \xi \leqslant h $, tem área $ f(\xi) $. Sejam $ B_1, M $ e $ B_2 $ as áreas das seções de $ S $ perpendiculares ao eixo dos $ x $ que passam respectivamente por $ (0, 0, 0), (h/2, 0, 0) $ e $ (h, 0, 0) $. Determine o volume de $ S $ em função de $ B_1, M $ e $ B_2 $.
		
		\begin{solution}
			Podemos encontrar o volume desejado $ v\coloneq v(S) $ integrando as áreas de seção do sólido entre $ 0 $ e $ h $, portanto
			\begin{align}
				v&=\int_{0}^{h}f(t)\dif t=\int_{0}^{h}a\,t^{2}+b\,t+c\dif t\nonumber\\
				&=\eval{\dfrac{1}{3}a\,t^{3}+\dfrac{1}{2}b\,t^{2}+c\,t}_{0}^{h}\nonumber\\
				v&=\dfrac{1}{3}a\,h^{3}+\dfrac{1}{2}b\,h^{2}+c\,h.\label{eq:int}
			\end{align}
			
			Para expressar o resultado em termos de $ B_1,M $ e $ B_2 $ devemos avaliar os pontos dados:
			\begin{gather}
%				\begin{numcases}
				B_1=f(0)=\eval{\del{a\,t^{2}+b\,t+c}}_{t=0}=c,\label{eq:cc} \\
				M=f\del{\dfrac{h}{2}}=a\del{\dfrac{h}{2}}^{2}+b\del{\dfrac{h}{2}}+c,\text{ e}\label{eq:c2}\\
				B_2=f(h)=a\,h^{2}+b\,h+c.\label{eq:c3}
%				\end{numcases}
			\end{gather}
			Fazendo $ B_2-2M $ e substituindo \ref{eq:cc}, temos
			\begin{equation}\label{eq:ca}
				B_2-2M=a\,\dfrac{h^{2}}{2}-B_1\implies a=\dfrac{2}{h^{2}}\del{B_2+B_1-2M}.
			\end{equation}
			Substituindo \ref{eq:ca} e \ref{eq:cc} em \ref{eq:c3}, temos
			\begin{equation}\label{eq:cb}
				B_2=\dfrac{2}{h^{2}}\del{B_2+B_1-2M}\,h^{2}+b\,h+B_1=2B_2+3B_1-4M+b\,h\implies b=\dfrac{1}{h}\del{4M-B_2-3B_1}.
			\end{equation}
		
			Dessa forma, temos o volume dado pelo sólido como sendo
			\begin{align*}
				v&=h\del{\dfrac{1}{3}\dfrac{2}{h^{2}}\del{B_2+B_1-2M}h^{2}+\dfrac{1}{2}\dfrac{1}{h}\del{4M-B_2-3B_1}h+B_1}\\
				&=h\del{\dfrac{2}{3}\del{B_2+B_1-2M}\cdot\dfrac{2}{2}+\dfrac{1}{2}\del{4M-B_2-3B_1}\cdot\dfrac{3}{3}+B_1}\\
				&=h\dfrac{1}{6}\del{4B_2+4B_1-8M+12M-3B_2-9B_1+6B_1}\\
				v&=\dfrac{h}{6}\del{B_1+4M+B_2}.
			\end{align*}
		
%			e, dessa forma, substituindo 
%			Pela definição da função temos
%			\[ v=\int_{0}^{h}a\,t^{2}+b\,t+c\dif t. \]
		\end{solution}
		
		\titledquestion{Grupo VII --- Q15}
		
		Seja $ f : \mathbb{R} \to \mathbb{R} $ contı́nua. Suponha que existem reais $ a_1, a_2, \ldots, a_p $ diferentes $ (p \geqslant 2) $, tais que $ f(a_k) =a_{k+1} $, se $ 1 \leqslant k \leqslant p-1 $ e $ f(a_p) = a_1 $. Prove que existe pelo menos um $ \bar{x} \in \mathbb{R} $ tal que $ f(\bar{x}) = \bar{x} $.
		
		\begin{solution}
			\paragraph{Nota:} essa resolução foi feita em conjunto com um colega do curso de matemática.
			
			Sejam ${i_1}, \ldots, i_p$ um rearranjo dos índices $1,\ldots,p$ de modo que $a_{i_{k}} \leqslant a_{i_{k+1}}$ para todo $k<p$.
		
				
		\begin{figure}[H]
			\centering
			
			
			\tikzset{every picture/.style={line width=0.75pt}} %set default line width to 0.75pt        
			
			\begin{tikzpicture}[x=0.75pt,y=0.75pt,yscale=-1,xscale=1]
				%uncomment if require: \path (0,300); %set diagram left start at 0, and has height of 300
				
				%Shape: Axis 2D [id:dp6481518428974078] 
				\draw  (53.5,200.1) -- (317,200.1)(79.85,7.5) -- (79.85,221.5) (310,195.1) -- (317,200.1) -- (310,205.1) (74.85,14.5) -- (79.85,7.5) -- (84.85,14.5) (99.85,195.1) -- (99.85,205.1)(119.85,195.1) -- (119.85,205.1)(139.85,195.1) -- (139.85,205.1)(159.85,195.1) -- (159.85,205.1)(179.85,195.1) -- (179.85,205.1)(199.85,195.1) -- (199.85,205.1)(219.85,195.1) -- (219.85,205.1)(239.85,195.1) -- (239.85,205.1)(259.85,195.1) -- (259.85,205.1)(279.85,195.1) -- (279.85,205.1)(299.85,195.1) -- (299.85,205.1)(59.85,195.1) -- (59.85,205.1)(74.85,180.1) -- (84.85,180.1)(74.85,160.1) -- (84.85,160.1)(74.85,140.1) -- (84.85,140.1)(74.85,120.1) -- (84.85,120.1)(74.85,100.1) -- (84.85,100.1)(74.85,80.1) -- (84.85,80.1)(74.85,60.1) -- (84.85,60.1)(74.85,40.1) -- (84.85,40.1) ;
				\draw   ;
				%Straight Lines [id:da34830364825233673] 
				\draw  [dash pattern={on 0.84pt off 2.51pt}]  (121,40) -- (120,200) ;
				%Straight Lines [id:da21301944065434042] 
				\draw  [dash pattern={on 0.84pt off 2.51pt}]  (241,39.33) -- (239.83,200.03) ;
				%Straight Lines [id:da38423440107926] 
				\draw  [dash pattern={on 0.84pt off 2.51pt}]  (79.83,39.97) -- (241,39.33) ;
				%Straight Lines [id:da28398829303809614] 
				\draw  [dash pattern={on 0.84pt off 2.51pt}]  (79.17,160.63) -- (240.33,160) ;
				%Shape: Circle [id:dp7463394320389365] 
				\draw  [fill={rgb, 255:red, 0; green, 0; blue, 0 }  ,fill opacity=1 ] (119,120.58) .. controls (119,119.89) and (119.56,119.33) .. (120.25,119.33) .. controls (120.94,119.33) and (121.5,119.89) .. (121.5,120.58) .. controls (121.5,121.27) and (120.94,121.83) .. (120.25,121.83) .. controls (119.56,121.83) and (119,121.27) .. (119,120.58) -- cycle ;
				%Shape: Circle [id:dp6684633384567282] 
				\draw  [fill={rgb, 255:red, 0; green, 0; blue, 0 }  ,fill opacity=1 ] (139,40.58) .. controls (139,39.89) and (139.56,39.33) .. (140.25,39.33) .. controls (140.94,39.33) and (141.5,39.89) .. (141.5,40.58) .. controls (141.5,41.27) and (140.94,41.83) .. (140.25,41.83) .. controls (139.56,41.83) and (139,41.27) .. (139,40.58) -- cycle ;
				%Shape: Circle [id:dp10454278311396448] 
				\draw  [fill={rgb, 255:red, 0; green, 0; blue, 0 }  ,fill opacity=1 ] (159,60.58) .. controls (159,59.89) and (159.56,59.33) .. (160.25,59.33) .. controls (160.94,59.33) and (161.5,59.89) .. (161.5,60.58) .. controls (161.5,61.27) and (160.94,61.83) .. (160.25,61.83) .. controls (159.56,61.83) and (159,61.27) .. (159,60.58) -- cycle ;
				%Shape: Circle [id:dp2963444429880262] 
				\draw  [fill={rgb, 255:red, 0; green, 0; blue, 0 }  ,fill opacity=1 ] (218,159.58) .. controls (218,158.89) and (218.56,158.33) .. (219.25,158.33) .. controls (219.94,158.33) and (220.5,158.89) .. (220.5,159.58) .. controls (220.5,160.27) and (219.94,160.83) .. (219.25,160.83) .. controls (218.56,160.83) and (218,160.27) .. (218,159.58) -- cycle ;
				%Shape: Circle [id:dp8857686056743208] 
				\draw  [fill={rgb, 255:red, 0; green, 0; blue, 0 }  ,fill opacity=1 ] (238,139.58) .. controls (238,138.89) and (238.56,138.33) .. (239.25,138.33) .. controls (239.94,138.33) and (240.5,138.89) .. (240.5,139.58) .. controls (240.5,140.27) and (239.94,140.83) .. (239.25,140.83) .. controls (238.56,140.83) and (238,140.27) .. (238,139.58) -- cycle ;
				%Curve Lines [id:da692684928373148] 
				\draw [color={rgb, 255:red, 255; green, 0; blue, 0 }  ,draw opacity=1 ]   (99,134) .. controls (103,145.33) and (109.5,143.33) .. (120.25,119.33) .. controls (131,95.33) and (128.21,57.87) .. (140.25,40.58) .. controls (152.29,23.29) and (159.67,59.33) .. (176.33,104.67) .. controls (193,150) and (208.17,221.33) .. (219.25,158.33) .. controls (230.33,95.33) and (225.67,182) .. (239.25,140.83) .. controls (252.83,99.67) and (264.25,175.83) .. (289.25,150.83) ;
				
				% Text Node
				\draw (114.21,205.36) node [anchor=north west][inner sep=0.75pt]  [font=\scriptsize]  {$a_{i_{1}}$};
				% Text Node
				\draw (132.21,205.03) node [anchor=north west][inner sep=0.75pt]  [font=\scriptsize]  {$a_{i_{2}}$};
				% Text Node
				\draw (154.21,205.36) node [anchor=north west][inner sep=0.75pt]  [font=\scriptsize]  {$a_{i_{3}}$};
				% Text Node
				\draw (185.94,208.38) node [anchor=north west][inner sep=0.75pt]  [font=\footnotesize]  {$\cdots $};
				% Text Node
				\draw (231.55,205.36) node [anchor=north west][inner sep=0.75pt]  [font=\scriptsize]  {$a_{i_{p}}$};
				% Text Node
				\draw (56.21,154.7) node [anchor=north west][inner sep=0.75pt]  [font=\scriptsize]  {$a_{i_{1}}$};
				% Text Node
				\draw (54.88,133.7) node [anchor=north west][inner sep=0.75pt]  [font=\scriptsize]  {$a_{i_{2}}$};
				% Text Node
				\draw (55.55,114.7) node [anchor=north west][inner sep=0.75pt]  [font=\scriptsize]  {$a_{i_{3}}$};
				% Text Node
				\draw (56,69.4) node [anchor=north west][inner sep=0.75pt]  [font=\footnotesize]  {$\vdots $};
				% Text Node
				\draw (56.21,31.36) node [anchor=north west][inner sep=0.75pt]  [font=\scriptsize]  {$a_{i_{p}}$};
				% Text Node
				\draw (316.67,202.07) node [anchor=north west][inner sep=0.75pt]    {$x$};
				% Text Node
				\draw (82,3.4) node [anchor=north west][inner sep=0.75pt]    {$y$};
				% Text Node
				\draw (250.67,107.07) node [anchor=north west][inner sep=0.75pt]  [color={rgb, 255:red, 252; green, 0; blue, 0 }  ,opacity=1 ]  {$y=f( x)$};
				
				
			\end{tikzpicture}
		\caption{Ilustração gráfica.}
		\end{figure}
		
%\end{exercise}
			
			Considere uma função $g\colon \mathbb{R} \longrightarrow\mathbb{R}$ dada por $g(x) \coloneqq f(x)-x$. Como $f$ é contínua, é claro que $g$ também é. Sejam $a,b\in\intcc{a_{i_1},a_{i_p}}$ tais que
			$$f(a) = \max\limits_{x\in\intcc{a_{i_1},a_{i_p}} } f(x) \quad\text{e}\quad f(b) = \min\limits_{x\in\intcc{a_{i_1},a_{i_p}} } f(x).$$
			Tais números existem pois $f$ é contínua e $\intcc{a_{i_1},a_{i_p}}$ é um intervalo fechado. 	Dessa forma, note que $f(a) \geqslant a_{i_p} \geqslant x$  e $f(b)\leqslant a_{i_1} \leqslant x$ para todo $x\in \intcc{a_{i_1},a_{i_p}}$. Ou seja: $g$ atinge tanto valores não-negativos ($g(a)\geqslant 0$) quanto não-positivos ($g(b)\leqslant 0$). 
			
			Supondo, sem perda de generalidade, que $a<b$, decorre do \textit{teorema do valor intermediário} que existe um $\bar x\in \intcc{a,b} \subseteq \intcc{a_{i_1},a_{i_p}}$ tal que $g(\bar x) = 0$. Dessa forma, $\bar x$ é tal que $0 =  f(\bar x) - \bar x$, i.e., $f(\bar x) = \bar x$.
			
			\hfill\qedsymbol
		\end{solution}
		
		\clearpage
		
		\titledquestion{Grupo VIII --- Q17}
		
		Provar que $ \dfrac{79}{160}\leqslant\displaystyle\int_{0}^{1/2}\sqrt{1-\xi^{4}}\dif \xi \leqslant \dfrac{79}{600}\sqrt{15}. $
		
		\begin{solution}
			Sabemos que
			\begin{align*}
				\dfrac{1}{\sqrt{1-0^{4}}}\leqslant&\dfrac{1}{\sqrt{1-\xi^{4}}}\leqslant \dfrac{1}{\sqrt{1-\del{\dfrac{1}{2}}^{4}}}\\
				\intertext{que não se altera caso multipliquemos por $ 1-\xi^{4} $}
				1-\xi^{4}\leqslant&\dfrac{1-\xi^{4}}{\sqrt{1-\xi^{4}}}\leqslant \dfrac{4}{\sqrt{15}}\del{1-\xi^{4}}\\
				\intertext{integrando $ \xi $ de $ 0 $ a $ 1/2 $, temos}
				\int_{0}^{1/2}1-\xi^{4}\dif \xi\leqslant&\int_{0}^{1/2}\dfrac{1-\xi^{4}}{\sqrt{1-\xi^{4}}}\dif \xi\leqslant\int_{0}^{1/2}\dfrac{4}{\sqrt{15}}\del{1-\xi^{4}}\dif \xi\\
				\eval{\xi-\dfrac{1}{5}\xi^{5}}_{0}^{1/2}\leqslant&\int_{0}^{1/2}\sqrt{1-\xi^{4}}\dif \xi\leqslant\dfrac{4}{\sqrt{15}}\eval{\del{\xi-\dfrac{1}{5}\xi^{5}}}_{0}^{1/2}\\
				\dfrac{1}{2}-\dfrac{1}{5}\del{\dfrac{1}{2}}^{5}\leqslant&\int_{0}^{1/2}\sqrt{1-\xi^{4}}\dif \xi\leqslant\dfrac{4}{\sqrt{15}}\del{\dfrac{1}{2}-\dfrac{1}{5}\del{\dfrac{1}{2}}^{5}}\\
				\dfrac{80-1}{160}\leqslant&\int_{0}^{1/2}\sqrt{1-\xi^{4}}\dif \xi\leqslant\dfrac{4\cdot\sqrt{15}}{15}\del{\dfrac{80-1}{160}}\\
				\dfrac{79}{160}\leqslant&\int_{0}^{1/2}\sqrt{1-\xi^{4}}\dif \xi\leqslant\dfrac{79}{600}\sqrt{15}.
			\end{align*}
		\hfill\qedsymbol
		\end{solution}
		
		\titledquestion{Grupo IX --- Q18}
		
		Prove que se $ f $ é integrável em $ \intcc{a, b} $, $ \int_{a}^{b}(f(x))^{2} \dif x = 0 $ e $ f $ é contı́nua em $ c \in \intcc{a,b} $, então $ f(c) = 0 $.
		
		\begin{solution}
			Sendo a integral, vulgarmente, uma soma de infinitos termos, ao tomarmos \[ \int_{a}^{b} (f(x))^{2}\dif x=\int_{a}^{b} |f(x)|^{2}\dif x=0, \]
			como $ |f|\geqslant 0 $, essa soma somente será nula se todos os termos forem nulos, dessa forma é garantido que, para $ c\in\intcc{a,b},f(c)=0 $.
		\end{solution}
		
		\titledquestion{Grupo X --- Q20}
		
		Suponha que $ S $ é uma esfera de raio $ R > 0 $ e	que $ u : S \to R $ é a função temperatura (num determinado instante $ T $ fixado), isto é $ u(p) $ é a temperatura do ponto $ p $ no instante $ T $. Suponha que $ u $ é contı́nua e mostre que existem pontos antı́podas que tem mesma temperatura.
		
		\begin{solution}
			\paragraph{Nota:} essa resolução foi feita em conjunto com um colega do curso de matemática.
			
			Dado um ponto $p \in S$, identifique por $-p \in S$ seu \textit{antípoda}. Considere
			\begin{equation*}
				\begin{array}{lccc}
					\Delta : & S & \longmapsto & \mathbb{R}\\
					& p & \longmapsto & u(p)-u(-p)
				\end{array}
				%			\function \Delta{S}{\R}{p}{u(p)-u(-p)}
			\end{equation*}
			Como $u$ é contínua, $\Delta$ obviamente também é. Restrinja $\Delta$ a um círculo máximo $C$, i.e. uma circunferência de raio $R$ inteiramente contida em $S$. Denote esta restrição de função por $\restrict{C}$. 
			
			Um círculo máximo $C$ contido em um plano $\alpha$ pode ser parametrizado por uma função $f_{\alpha} : [0,2\pi] \longrightarrow C$ contínua, dado que basta transladar e inclinar o conjunto $\set{(R\cos(t),R\sin(t), 0) \mid t\in \intcc{0,2\pi}}$.
			
			Dessa forma, $ u\restrict{C} \circ f_{\alpha} \colon [0,2\pi] \longrightarrow \mathbb{R}$ é uma função contínua. Dado $[0,2\pi]$ é um intervalo fechado, existem instantes  $t_{\text{max}},t_{\text{min}}\in [0,2\pi]$ tais que:
			\begin{itemize}[label=\textbullet]
				\item $(u \restrict{C} \circ f_{\alpha})(t_{\text{max}})$ é um valor de máximo.
				\item $(u \restrict{C} \circ f_{\alpha})(t_{\text{min}})$ é um valor de mínimo.
			\end{itemize}
			Note que $u(f_\alpha(t_{\text{max}}))\geqslant u(-f_\alpha(t_{\text{max}}))$ e $u(f_\alpha(t_{\text{min}}))\geqslant u(-f_\alpha(t_{\text{min}}))$. Ou seja $\Delta\restrict{C}$ possui valores positivos e negativos. Supondo sem perda de generalidade que $t_{\text{max}} < t_{\text{min}}$, o \textit{teorema do valor intermediário} garante que existe um ponto $t\in [t_{\text{max}},t_{\text{min}}]\subset [0,2\pi]$ tal que $(\Delta \restrict{C} \circ f_{\alpha})(t) = 0$. No ponto $p\coloneqq  f_{\alpha}(t)$,  por construção, $u(p)=u(-p)$. 
			
			Dessa forma, em cada círculo máximo $C\subset S$, existe um par de pontos antipodais com mesma temperatura.
		\end{solution}
		
%		\titledquestion{Grupo Festa --- Q21}
%		Sejam $ n\in\mathbb{N} $ e $ f\colon\intcc{a,b}\longrightarrow \intoo{0,+\infty} $ integrável.
%		
%		\begin{enumerate}[label=(\roman*)]
%			\item\label{item:qi} Prove que existe uma partição $ a=c_0<c_1<c_2<\cdots<c_n=b $ de $ \intcc{a,b} $ tal que, para todo $ j\in\set{1,\ldots,n} $, tem-se $ \int_{c_{j-1}}^{c_j}f(x)\dif x=\dfrac{1}{n}\int_{a}^{b}f(x)\dif x. $
%		
%			\item Prove que existe uma única partição de $ \intcc{a,b} $ com a propriedade enunciada em \ref{item:qi}.
%		\end{enumerate}
%	
%		\begin{solution}
%			Pelo teorema 3.15 temos que a integral no intervalo completo pode ser escrita como
%			\[ \int_{a}^{b}f(x)\dif x=f(c)(b-a), \]
%			para algum $ c\in\intcc{a,b} $.
%			
%			De maneira análoga, defina as integrais de cada uma das partições
%			\[ S_j \coloneq \int_{c_{j-1}}^{c_{j}}f(x)\dif x=f(c_i)(c_j-c_{j-1}), \]
%			para $ c_i\in\intcc{c_{j-1},c_j} $.
%			
%			Seja $ f(c'_j)(c_j-c_{j-1}) $ um produto da forma $ f(c)\Delta_j $, onde $ f(c) $ é o mesmo da aplicação de 3.15 à integral de $ f(x) $ entre $ a $ e $ b $ e $ \Delta_j $ é um produto $ \lambda_j(c_j-c_{j-1}) $, com $ \lambda_j=1/m_j, m_j\cdot f(c'_j)=f(c) $. É notável que $ \Delta_j/\lambda_j=c_j-c_{j-1}=f(c'_j)/f(c) $ possui somente uma solução, pois $ a $ e $ b $ (extremos) são fixados (a solução única para $ c_1=f(c'_1)/f(c)+c $ implica na solução única para $ c_2 $, etc.).
%			
%			Somando todos os $ n $ termos $ S_j $, temos
%			\[ \sum_{j=0}^{n}f(c'_j)(c_j-c_{j-1})=\sum_{j=0}^{n}f(c)\Delta_j= \]
%		\end{solution}
		
	\end{questions}
\end{document}