\documentclass{IMTexam}

\usepackage[enums]{IMTtikz}
\usepackage{booktabs}
\usepackage{stackengine,collcell}
\let\endminwd\relax
%\newcolumntype{L}[1]{>{\collectcell\xminwd l{#1}$}l<{$\endminwd\endcollectcell}}
\newcolumntype{C}[1]{>{\collectcell\xminwd c{#1}$}c<{$\endminwd\endcollectcell}}
%\newcolumntype{R}[1]{>{\collectcell\xminwd r{#1}$}r<{$\endminwd\endcollectcell}}
\def\minwd#1#2#3\endminwd{\stackengine{0pt}{#3}{\rule{#2}{0pt}}{O}{#1}{F}{F}{L}}
\newcommand\xminwd[1]{\minwd#1}

\givecredits
\author{Isabella B. Amaral}
\USPN{11810773}
\lecture{Matemática II}
\examname{Prova I}
\hwtype{Resolução}
\lcode{}
\date{20 de maio}

\newtheorem{definition}{Definição}
\newtheorem{theorem}{Teorema}[question]
\newtheorem{corollary}{Corolário}[theorem]
\newtheorem{lemma}[theorem]{Lema}
\let\oldemptyset\emptyset
\let\emptyset\varnothing
\newcommand\restrict[1]{{% we make the whole thing an ordinary symbol
        \left.\kern-\nulldelimiterspace % automatically resize the bar with \right
        % the function
        \vphantom{\big|} % pretend it's a little taller at normal size
        \right|_{#1} % this is the delimiter
}}
\DeclarePairedDelimiter\ceil{\lceil}{\rceil}
\DeclarePairedDelimiter\floor{\lfloor}{\rfloor}

\newcommand{\decSep}{,}
\newcommand{\decNum}[3]{\rlap{\ensuremath{\phantom{#1\mathord{\decSep}#2}\overline{\phantom{#3}}}}\num{#1.#2#3}}
%\newcommand{\decSI }[4]{\rlap{\ensuremath{\phantom{#1\mathord{\decSep}#2}\overline{\phantom{#3}}}}\SI{#1.#2#3}{#4}}


\begin{document}

    \maketitle

    \begin{questions}
        \question Se o seu número USP for par demonstre que $ \cos 1 $ é
        irracional, mas se o seu número USP for ímpar demonstre que $ \sin 1 $ é
        irracional.

        \begin{solution}
            Sabemos, por 7.1, uma função $ f $ pode ser aproximada unicamente por um polinômio em
            torno de um ponto $ a $ por
            \[ f(x) = T(f, a)(x) = f(a) + \dfrac{f'(a)}{1!}(x - a) +
            \dfrac{f^{\prime\prime}(a)}{2!}(x - a)^2 + ... = \sum_{n = 0}^\infty
            \dfrac{f^{(n)}}{n!}(x - a)^n. \]
            Dessa forma podemos fazer
            \begin{align*}
                \sin 1 = T(\sin, 0)(1) &= \sin 0 +
                \dfrac{\sin'(0)}{1!}(x - 0) +
                \cancel{\dfrac{\sin^{\prime\prime}(0)}{2!}(x - 0)^2} + \cdots\\
                &= 1 - \dfrac{1}{3!} + \dfrac{1}{5!} - \cdots,
            \end{align*}
            portanto, supondo que $ \alpha=\sin 1 $ é da forma $ a/b $, onde $
            a,b\in\mathbb{Z} $, devemos ter $b!\alpha$ como um inteiro. Vamos
            separá-lo em duas porções $b!\alpha = C+D$. Defina
            \[s=\begin{cases}
                b-1,&\textup{para $b$ ímpar}\\
                b,&\textup{para $b$ par}
            \end{cases}\]
            fazemos, então
            \[ C =  1 - \dfrac{1}{3!} + \cdots + (-1)^{-(s+1)/2}\dfrac{b!}{s!} \]
            de tal forma que é claramente formado por inteiros e, portanto, a
            porção restante, $D$, também deve ser inteira.
            Porém verificamos que
            \[ D = \begin{cases}
                (-1)^{-(s+1)/2+1}\del{\dfrac{1}{(b+1)}-\dfrac{1}{(b+1)(b+3)}+\cdots},&\textup{para $b$ par}\\
                (-1)^{-(s+1)/2+1}\del{\dfrac{1}{(b+2)}-\dfrac{1}{(b+2)(b+4)}+\cdots},&\textup{para $b$ ímpar}
            \end{cases}\]
            e, sendo o segundo caso estritamente menor que o primeiro, temos que as somas são limitadas por
            \[ \dfrac{1}{(b+1)}-\dfrac{1}{(b+1)(b+3)}+\cdots <
            \dfrac{1}{(b+1)}+\dfrac{1}{(b+1)(b+3)(b+5)}+\cdots <
            \sum_{n=1}^\infty (b+1)^{-1} \]
            que é uma série geométrica convergente.
            Como essa série converge para
            \[ \sum_{n=1}^\infty (b+1)^{-1} = \dfrac{1/(b+1)}{1-1/(b+1)} = \dfrac{1}{b} < 1 \]
            e, como os termos são não nulos, temos uma contradição, pois $D\notin \mathbb{Z}$.

            \hfill\qedsymbol
        \end{solution}
        \clearpage

        \question Suponha que $ f \colon (−1, 1) \longrightarrow \mathbb{R} $
        e $ g \colon (−1, 1) \longrightarrow \mathbb{R} $ são funções
        cinco vezes deriváveis, $ f $ é ímpar,
        \[ \lim\limits_{x\to0} \dfrac{f(x)−\sin(x)}{x^5} = 0 \]
        e $ T_3(g;0)(x) = 1 − x+x^3$.

        Suponha que $ h(x) = f (x)g(x) $ satisfaz $ h^{(5)}(0) = 1 $
        e calcule $ T_4(g; 0)(x) $.

        \begin{solution}
            Pelo limite, temos que $f(x)$ é bem aproximado por um seno, até um
            polinômio de quinto grau, de tal forma que $f(0)=0, f'(0)=1,
            f^{\prime\prime}(0)=0, f^{(3)}(0)=-1, f^{(4)}(0)=0, f^{(5)}(0)=1$, e
            portanto, temos
            \begin{align*}
            h^{(5)}(0)&= \cancelto{1}{f^{(5)}g(0)}+ \cancel{5f^{(4)}(0)g'(0)}+
            \cancelto{-1}{f^{(3)}}10g^{\prime\prime}(0)+
            \cancel{10f^{\prime\prime}(0)g^{(3)}(0)}+
            \cancelto{1}{f'(0)}5g^{(4)}(0)+ \cancel{f(0)\,g^{(5)}(0)}\\
            1&=1-10g^{\prime\prime}(0)+5g^{(4)}(0)\\
            \intertext{pelo formato de $T_3(g;0)(x)$ notamos que $g^{\prime\prime}(0)=0$, portanto}
            0&=g^{(4)}(0),
            \end{align*}
            e, dessa forma, $T_4(g; 0)(x)=T_3(g;0)(x) = 1 − x+x^3$.
        \end{solution}

        \question Calcule as integrais abaixo:

        \begin{enumerate}[label=(\roman*)]
            \item \[ \int \dfrac{3x^2+2}{(x-1)(x^2+4)^2}\dif x \]
            \begin{solution}
                Por decomposição em frações parciais, temos que
                \begin{align*}
                    \dfrac{3x^2+2}{(x-1)(x^2+4)^2} &=
                    \dfrac{A}{x-1}+\dfrac{Bx+C}{x^2+4}+\dfrac{Dx+E}{(x^2+4)^2}\\
                    &=\dfrac{A(x^2+4)^2+(Bx+C)(x-1)(x^2+4)+(Dx+E)(x-1)}{(x-1)(x^2+4)^2}\\
                    %&=\dfrac{(A+B)x^4+(C-B)x^3+(8A+4B-C+D)x^2+(4C-4B+E-D)x+(16A-4C-E)}{(x-1)(x^2+4)^2}
                \end{align*}
                e, comparando coeficientes, temos
                \[ \begin{cases}
                    A+B=0\implies A=-B\\
                    C-B=0\implies B=C\\
                    8A+4B-C+D=3\implies -5C+D=3\\
                    4C-4B+E-D=0\implies D=E\\
                    16A-4C-E=2\implies -20C-D=2
                \end{cases} \]
                o que implica em $ B=C=-A=-1/5 $ e $ D=E=2 $.
                Agora temos um problema mais simples
                \begin{align*}
                    \int \dfrac{3x^2+2}{(x-1)(x^2+4)^2}\dif
                    x&=\int\dfrac{1/5}{x-1}+\dfrac{-x/5-1/5}{x^2+4}+\dfrac{2x+2}{(x^2+4)^2}\dif
                    x\\
                    &=\dfrac{1}{5}\underbrace{\int\dfrac{1}{x-1}\dif
                    x}_{(1)}-\dfrac{1}{5}\underbrace{\int\dfrac{x+1}{x^2+4}\dif
                    x}_{(2)}+2\underbrace{\int\dfrac{x+1}{(x^2+4)^2}\dif
                    x}_{(3)}
                \end{align*}
                Substituindo $ u = x-1 \implies \dif u = \dif x $
                notamos que a integral (1) se trata de um $ \ln x $ (por definição).
                Para a segunda integral, podemos fazer
                \begin{align*}
                    \int \dfrac{x+1}{x^2+4}\dif x&=\int \dfrac{(x/2)/2+1/4}{\del{x/2}^2+1}\dif x\\
                    \intertext{tomando $\tan t = x/2 \implies \dif x = \sec^2 t \dif t$}
                    &=\int \dfrac{\tan t/2+1/4}{\tan^2 t+1}\sec^2 t \dif t\\
                    &=\int \dfrac{1}{2}\dfrac{\sin t}{\cos t}\dif t+\int\dfrac{1}{4}\dif t\\
                    \intertext{seja $v = \cos t \implies \dif v = -\sin t \dif t$}
                    &=\int \dfrac{1}{2}\dfrac{-\dif v}{v}+\dfrac{\arctan (x/2)}{4}\\
                    &=-\dfrac{1}{2}\ln \cos\arctan (x/2)+\dfrac{\arctan (x/2)}{4}.
                \end{align*}
                Pela relação fundamental da trigonometria temos que
                \[ \sin^2\theta+\cos^2\theta = 1\implies \cos^2 \theta=\del{\tan^2 \theta+1}^{-1} \]
                substituindo $\theta = \arctan \alpha$ temos
                \begin{equation}\label{eq:cosarctan}
                    \cos\arctan \alpha=\del{\alpha^2 + 1}^{-1/2}.
                \end{equation}
                Dessa forma a segunda integral iguala
                \[ \int \dfrac{x+1}{x^2+4}\dif x=\dfrac{1}{4}\del{\ln \del{(x/2)^2+1}+\arctan (x/2)} \]
                Por semelhança, notamos que a substituição trigonométrica na terceira integral nos dá
                \begin{align*}
                    \int\dfrac{\tan t/8+1/16}{(\tan^2 t+1)^2}\sec^2 t\dif
                    t&=\dfrac{1}{8}\int\dfrac{\tan t}{\sec^2 t}\dif t+
                    \dfrac{1}{16}\int\dfrac{1}{\sec^2 t} \dif t\\
                    &=\dfrac{1}{8}\int\underbrace{\sin t}_{=w} \cos t\dif t+ \dfrac{1}{16}\int \cos^2 t \dif t\\
                    &=\dfrac{1}{8}\int w \dif w +\dfrac{1}{32}\int 1 + \cos \underbrace{2t}_{=x} \dif t\\
                    &=\dfrac{1}{16} w^2 +\dfrac{1}{32}\del{t+\int \cos x \dif x}\\
                    \int\dfrac{x+1}{(x^2+4)^2}\dif x&=\dfrac{1}{16}\del{\sin^2
                    (x/2)+\dfrac{1}{2}\sin x +\dfrac{1}{4}x}
                \end{align*}
                Dessa forma, adotando que $C$ é uma constante arbitrária, temos:
                \[ \begin{split}
                    \int \dfrac{3x^2+2}{(x-1)(x^2+4)^2}\dif x=
                    \dfrac{1}{5}\ln\del{\dfrac{x-1}{\del{x^2/4+1}^{1/4}}}+
                    &\dfrac{1}{20}\arctan\del{\dfrac{x}{2}}+\\
                    \dfrac{1}{8}&\del{\sin^2\del{\dfrac{x}{2}}+\dfrac{1}{2}\sin x+\dfrac{1}{4}x} + C
                \end{split}
                 \]
            \end{solution}
            \item \[ \int \dfrac{1}{1+ a\sin x}\dif x\textup{, em que }0<a<1 \]
            \begin{solution}
                Fazendo $a\sin x = \tan^2 t \implies a\cos x \dif x = 2\tan t\sec^2 t\dif t$, portanto
                \begin{align*}
                    \int \dfrac{1}{1+ a\sin x}\dif x&=\int \dfrac{1}{1+\tan^2
                    t}\dfrac{2\tan t\sec^2 t}{a\cos t}\dif t=\int \dfrac{2\sin
                    t}{a\cos^2 t}\dif t\\
                    \intertext{fazendo $u = \cos t\implies \dif u = - \sin t \dif t$}
                    &=-\dfrac{2}{a}\int
                    \dfrac{\dif u}{u^2}=\dfrac{2}{au}\\
                    \intertext{adotando que $C$ é uma constante arbitrária, temos}
                    \int \dfrac{1}{1+ a\sin x}\dif x&=\dfrac{2}{a\cos
                    \sqrt{\arccos \del{a\sin x}}} + C
                \end{align*}
            \end{solution}
            \item \[ \int \dfrac{\sqrt{4-x-x^2}}{x^2}\dif x \]
            \item \[ \int \dfrac{\arctan{\sqrt{x}}}{(1+x)\sqrt{x}}\dif x \]
            \begin{solution}
                Primeiro, precisamos de alguns resultados parciais, como
                \begin{align*}
                    f\circ f^{-1}(x)=x\implies \dod{}{x}f^{-1}&=\dfrac{1}{\dod{}{u}f(u)},\quad u=f^-1\\
                    \implies \dod{}{x}\arctan(x)&=\dfrac{1}{\dod{}{u}\tan u},\quad u=\tan^{-1}\\
                        &=\dfrac{1}{\sec^2\tan^{-1}x}\\
                    \intertext{substituindo \ref{eq:cosarctan}}
                        \dod{}{x}\arctan(x)&=\cos^2\tan^{-1}x=\dfrac{1}{1+x^2}.
                \end{align*}
                Dessa forma, substituindo $ u = \arctan \sqrt x\implies \dif u = (1+x)^{-1}\od{}{x}\sqrt x = \dif x/(2(x+1)\sqrt x) $, temos
                \begin{align*}
                    \int \dfrac{\arctan{\sqrt{x}}}{(1+x)\sqrt{x}}\dif x&=\int 2u\dif u=u^2\\
                    \intertext{adotando que $C$ é uma constante arbitrária, temos}
                    \int \dfrac{\arctan{\sqrt{x}}}{(1+x)\sqrt{x}}\dif x&=\del{\arctan\sqrt x}^2 + C
                \end{align*}
            \end{solution}
        \end{enumerate}

        \question Justifique suas afirmações e cálculos.

        \begin{enumerate}[label=(\roman*)]
            \item Prove que a equação $ \cos x = x $ tem apenas uma solução
            positiva.

            \begin{solution}
                Sendo $f(x)=\cos x-x$ segue que $f'(x)=\sin x - 1\leqslant 0$,
                de tal forma que sempre $f(x)$ é estritamente decrescente e,
                dessa forma, sendo $f(0)=1>-\pi/2=f(\pi/2)$, pelo teorema do
                valor intermediário deve possuir uma solução única e positiva.
            \end{solution}
            \item Use que $ \num{1.7320} < \sqrt{3} < \num{1.7321} $
            e o polinômio de Taylor de ordem... de
            $ \cos x $ em zero para provar que $ r = \sqrt{3} − 1 $
            satisfaz $ \envert{\cos r − r} < \num{0.012} $.

            \begin{solution}
                Sendo a derivação da função $\cos x$ ``periódica", i.e.
                \begin{center}
                    \begin{tabular}{lr}\toprule
                        $f'(0)=$ & $\eval{\sbr{-\sin x}}_{x=0}=0$ \\
                        $f^{\prime\prime}(0)=$ & $\eval{\sbr{-\cos x}}_{x=0}=-1$ \\
                        $f^{(3)}(0)=$ & $\eval{\sbr{\sin x}}_{x=0}=0$ \\
                        $f^{(4)}(0)=$ & $\eval{\sbr{\cos x}}_{x=0}=1$ \\
                        \bottomrule
                    \end{tabular}
                \end{center}
                podemos tomar a expansão de Taylor de ordem 4
                \[ T_4(\cos; 0)(x) = 1-\dfrac{x^2}{2!}+\dfrac{x^4}{4!}, \]
                e, substituindo o valor desejado, temos:
                \begin{align*}
                    T_4(\cos; 0)(r = \sqrt{3} − 1) &= 1-\dfrac{\del{\sqrt{3} − 1}^2}{2!}+\dfrac{\del{\sqrt{3} − 1}^4}{4!}\\
                    &=1-\dfrac{3+1-2\sqrt 3}{2!}+\dfrac{\del{4-2\sqrt 3}^2}{4!}\\
                    &=\del{\sqrt 3 - 1}+\dfrac{16-16\sqrt 3 + 12}{4!}\\
                    T_4(\cos; 0)(r)&=\del{\sqrt 3 - 1}+\dfrac{7-4\sqrt 3}{6}
                \end{align*}
                de tal forma que a expressão desejada nos deixa com
                \[ \cos r - r \approx \del{\sqrt 3 - 1}+\dfrac{7-4\sqrt 3}{6}-\del{\sqrt 3 - 1}=\dfrac{7-4\sqrt 3}{6}, \]
                portanto, temos
                \begin{align*}
                    \num{1.7320} &< \sqrt{3} < \num{1.7321}\\
                    -\num{1.7320} &> -\sqrt{3} > -\num{1.7321}\\
                    \dfrac{7-4\cdot \num{1.7320}}{6}&> \dfrac{7-4\sqrt 3}{6} > \dfrac{7-4\cdot \num{1.7321}}{6}\\
                    \num{0.012}&> \dfrac{7-4\sqrt 3}{6} > \decNum{0}{0119}{3}\\
                    \implies \envert{\dfrac{7-4\sqrt 3}{6}}=\envert{\cos r − r}&< \num{0.012}.
                \end{align*}

                \hfill\qedsymbol
            \end{solution}
            \item Decida se $ r $ é maior ou menor do que a raiz
            positiva da equação $ \cos x = x $.

            \begin{solution}
                Sabemos que a função $f(x)=\cos x - x$ é estritamente
                decrescente, dessa forma, dado que $\num{0.012}>\cos
                r-r=f(r)>\decNum{0}{0119}{3}$,
                $r$ é maior que a raíz positiva de $f(x)=0$.
            \end{solution}
        \end{enumerate}

        \question Use as duas primeiras derivadas não nulas de
        $ f(x) = \sin(x^2) $ em zero e obtenha uma
        aproximação de $ \int f (x) \dif x$ com o uso do polinômio de
        Taylor correspondente.

        Apresente a melhor estimativa de erro que conseguir para sua
        aproximação.

        \begin{solution}
            Derivando a função $f(x)$, temos
            \begin{center}
                \begin{tabular}{lr}\toprule
                    $f'(0)=$ & $\eval{\sbr{2x\cos(x^2)}}_{x=0}=0$\\
                    $f^{\prime\prime}(0)=$ & $\eval{\sbr{2\cos(x^2)-4x^2\sin (x^2)}}_{x=0}=2$ \\
                    $f^{(3)}(0)=$ & $\eval{\sbr{-4x\del{3\sin(x^2)+x^2\cos(x^2)}}}_{x=0}=0$ \\
                    $f^{(4)}(0)=$ & $\eval{\sbr{4\del{-3\sin(x^2)+15x^2\cos (x^2)+8x^4\sin(x^2)}}}_{x=0}=-12$\\
                    \bottomrule
                \end{tabular}
            \end{center}
            portanto, temos a integral da expansão de Taylor
            \[ \int T_4(f;0)(x)\dif x=\int \dfrac{2}{2!}x^2+\dfrac{-12}{4!}x^4\dif x=\dfrac{1}{3}x^3-\dfrac{1}{2}\dfrac{1}{5}x^5+C,\quad C\in\mathbb{R}, \]
            e, conforme a aproximação seria melhorada usando $T_5(f;0)(x)$, temos que
            \[ f(x)= \sum_{n = 0}^\infty \dfrac{f^{(n)}}{n!}(x - a)^n=
            T_4(f;0)(x) + \cdots\]
            De tal forma que, utilizando o próximo termo não nulo da série, i.e.
            \begin{align*}
                f^{(6)}(0)&=\eval{\dod[2]{}{x}\sbr{4\del{-3\sin(x^2)+15x^2\cos (x^2)+8x^4\sin(x^2)}}}_{x=0}\\
                &=\eval{\dod{}{x}\sbr{4\del{-3\cdot 2x\cos(x^2)+15\del{2x\cos (x^2)+x^2(-2x\sin(x^2))}+8\del{4x^3\sin(x^2)+x^4\cdot 2x\cos(x^2)}}}}_{x=0}\\
                &=\eval{\dod{}{x}\sbr{8\del{12x\cos (x^2)+x^3\sin(x^2)+8x^5\cos(x^2)}}}_{x=0}\\
                &=\eval{\sbr{8\del{12\del{\cos (x^2)-x\cdot 2x\sin(x^2)}+3x^2\sin(x^2)+x^3\cdot 2x\cos(x^2)+8\del{5x^4\cos(x^2)-x^5\cdot 2x\sin(x^2)}}}}_{x=0}\\
                f^{(6)}(0)&=\eval{\sbr{8\del{12\cos (x^2)-9x^2\sin(x^2)+42x^4\cos(x^2)-16x^6\sin(x^2)}}}_{x=0}=48
            \end{align*}
            podemos estimar um erro de
            \[\int f(x)-T_4(f;0)(x)\dif x \approx \int \dfrac{f^{(6)}(0)}{6!}x^6 \dif x=\int \dfrac{48}{6!}x^6\dif x=\dfrac{48}{7!}x^7+C,\quad C\in\mathbb{R}. \]

        \end{solution}
    \end{questions}
\end{document}
