\documentclass[english]{IMTexam}

\usepackage[enums,english]{IMTtikz}
\usepackage{polynom}

\givecredits
\author{Isabella B. Amaral}
%\USPN{118010773}
\lecture{Number Theory}
\examname{Problem Set \#1}
\hwtype{Solutions}
\lcode{}
%\date{}

\newtheorem{definition}{Definition}
\newtheorem{theorem}{Theorem}[question]
\newtheorem{corollary}{Corolary}[theorem]
\newtheorem{lemma}[theorem]{Lemma}
\let\oldemptyset\emptyset
\let\emptyset\varnothing
\newcommand\restrict[1]{{% we make the whole thing an ordinary symbol
		\left.\kern-\nulldelimiterspace % automatically resize the bar with \right
		% the function
		\vphantom{\big|} % pretend it's a little taller at normal size
		\right|_{#1} % this is the delimiter
}}

\begin{document}
	
	\maketitle	
	
	\begin{questions}
		\question\label{ques:IR:1.23} Suppose that $ a^{2} + b^{2} = c^{2} $ with $ a, b, c\in \mathbb{Z} $. For example, $ 3^{2} + 4^{2} = 5^{2} $ and $ 5^{2} + 12^{2} = 13^{2} $. Assume that $ (a, b) = (b, c) = (c, a) = 1 $. Prove that there exist integers $ u $ and $ v $ such that $ c - b = 2u^{2} $ and $ c + b = 2v^{2} $ and $ (u, v) = 1 $ (there is no loss in generality in assuming that $ b $ and $ c $ are odd and that $ a $ is even). Consequently $ a = 2uv, b = v^{2} - u^{2} $, and $ c = v^{2} + u^{2} $. Conversely show that if $ u $ and $ v $ are given, then the three numbers $ a, b $, and $ c $ given by these formulas satisfy $ a^{2} + b^{2} = c^{2} $.
		
		\begin{solution}
			Assuming $ b,c $ odd and $ a=2uv $, such that $ u,v\in\mathbb{N} $, then we have
			\[ a^{2}=(2u\,v)^{2}=c^{2}-b^{2}=(c-b)(c+b)=(2u^{2})(2v^{2}) \]
			such that $ c-b=2u^{2} $ and $ c+b=2v^{2} $.
			
			Then, if $ u $ and $ v $ are given, we have $ a=2u\,v $ and then $ 2u^{2}+2v^{2}=2c-b+b=2\del{u^{2}+v^{2}}\implies c=u^{2}+v^{2} $ and also $ b=v^{2}-u^{2} $ (by the same logic).
			
			Taking the squares of $ a$ and $ b $, summing them up and then subtracting $ c^{2} $ we have
			\[ \del{2u\,v}^{2}+\del{v^{2}-u^{2}}^{2}-\del{u^{2}+v^{2}}^{2}=4u^{2}\,v^{2}+\del{v^{4}-2u^{2}\,v^{2}+u^{4}}-\del{u^{4}+2u^{2}\,v^{2}+v^{4}}=0\implies a^{2}+b^{2}=c^{2}. \]
			
			\hfill\qedsymbol
		\end{solution}
		
		\question\label{ques:IR:1.25} If $ a^{n} - 1 $ is a prime, show that $ a = 2 $ and that $ n $ is a prime. Primes of the form $ 2^{p} - 1 $ are called Mersenne primes. For example, $ 2^{3} - 1 = 7 $ and $ 2^{5} - 1 = 31 $. It is not known if there are infinitely many Mersenne primes.
		
		\begin{solution}
			By the factorization
			\[ a^{n}-1=(a-1)\sum_{k=0}^{n-1}a^{k}=p, \]
			$ p $ must have at least $ 2 $ divisors, and if $ a\neq 2  $ then $ p $ isn't prime.
			
			So we have a number of the form $ 2^{n}-1 $ which must be prime. Suppose $ n=\alpha\,\beta,\alpha,\beta\in\mathbb{N} $ (i.e. $ n $ is a composite number), then $ 2^{\alpha\,\beta}-1=p\implies 2=\sqrt[\alpha]{p+1}\sqrt[\beta]{p+1} $ which either (1) implies (without loss of generality) that $ \alpha=1 $ and $ \beta $ is a prime number or it implies that (2) the number $ 2 $ is composite, which is absurd.
			
			So then we must have $ 2^{n}-1 $ with $ n $ being a prime number, as we'd like to demonstrate.
		\end{solution}
		
		\question\label{ques:IR:1.26} If $ a^{n} + 1 $ is a prime, show that $ a $ is even and that $ n $ is a power of $ 2 $. Primes of the form $ 2^{2^{t}} + 1 $ are called Fermat primes. For example, $ 2^{2^{1}} + 1 = 5 $ and $ 2^{2^{2}} + 1 = 17 $. It is not known if there are infinitely many Fermat primes.
		
		\begin{solution}
			If $ a^{n}+1 $ is a prime then $ a $ must be even for if $ a $ were any odd number then $ a^{n} $ would also be odd, thus $ a^{n}+1 $ would be even. As $ 3^{1}+1=4>2 $ is the least value for this expression it couldn't be prime.
			
			So we know that we must have $ a=2m,m\in\mathbb{N} $. Suppose, then, that $ n=\alpha\,\beta,\alpha\neq\beta $, thus we'd have the factorization $ 2m=\sqrt[\alpha]{p-1}\sqrt[\beta]{p-1} $ and, without loss of generality, we could set $ 2=\sqrt[\alpha]{p-1} $ and $ m=\sqrt[\beta]{p-1} $, but then we'd fall in a contradiction, as $ \del{2m}^{n}=2^{\alpha}\,m^{\beta} $. Thus we conclude that $ \alpha $ must equal $ \beta $.
			
			Now we know that $ n=\gamma^{k} $, where $ \gamma $ is prime as it cannot have more than two divisors. Suppose that $ \gamma $ is an odd number (any prime $ >2 $). As $ \gamma $ is an odd power we have the expansion
			\[ a^{\gamma}+1=(a+1)\sum_{k=0}^{\gamma}(-1)^{k+1} a^{k} \]
			which is a contradiction, as $ a^{\gamma}+1 $ should be prime. Thus, $ \gamma $ must be even and, as such, must be $ 2 $ (for it must also be prime).
			
			Then we've concluded that $ a=2m $ and $ n=2^{t} $ so that we have $ (2m)^{2^{t}}+1=p $ being a prime number.
		\end{solution}
		
		\question\label{ques:IR:1.30} Prove that $ 1/2+1/3+\cdots+1/n $ is not an integer.
		
		\begin{solution}
			Assume $ 1/2+1/3+\cdots+1/n=a,a\in\mathbb{Z} $.
			
			Adopting the summation notation, we have that
			\[ \sum_{k=2}^{n}\dfrac{1}{k}=\sum_{k=2}^{n}\dfrac{n!/k}{n!} \]
			so that for $ \sum_{k=2}^{n}1/k $ to be an integer we must have $ n!\mid\sum_{k=2}^{n}n!/k $. By the lemma that every integer $ a $ can be written as $ qb+r $ we have that
			\[ \sum_{k=2}^{n}\dfrac{1}{k}=a=q\,n!+r\implies \sum_{k=2}^{n}\dfrac{n!}{k}=\sum_{k=2}^{n}a_k-r_k=q\,n!. \]
			For $ n!\mid\sum_{k=2}^{n}n!/k $ to be true we must have $ n!|r $, but as each term $ n!/k<n! $, then this residue must be non-zero:
			\[ \sum_{k=2}^{n}r_k=\sum_{k=2}^{n}\del{1-1/k}=q \]
			notice that $ r_k $ is simply the opposite of what lacks for $ n!/k $ to be divisible by $ n! $.\footnote{i.e. $ \del{n!/k}/n! =\del{n!\del{1/k+1-1}}/n! =1-\underbrace{\del{1-1/k}}_{r_k} $.}
			
			But as we evaluate the sum, we notice that
			\[ \sum_{k=2}^{n}\del{1-\dfrac{1}{k}}=\dfrac{(n+2)(n-1)}{2}-\underbrace{\sum_{k=2}^{n}\dfrac{1}{k}}_{\text{this should be divisible by $ n! $}}. \]
			but as $ (n-1)/n! =1/n(n-2)! $ and $ (n+2)/2<n(n-2)! $ the sum cannot be divisible by $ n! $. Contradiction!
			
			Thus $ 1/2+1/3+\cdots+1/n $ cannot be an integer, as we'd like to show.
		\end{solution}
		
		\question\label{ques:IR:2.4} If $ a $ is a nonzero integer, then for $ n > m $ show that $ (a^{2^{n}} + 1, a^{2^{m}} + 1) = 1 $ or $ 2 $ depending on whether $ a $ is odd or even. (Hint: If $ p $ is an odd prime and $ p|a^{2^{m}} + 1 $, then $ p|a^{2^{n}} - 1 $ for $ n > m $.)
		
		\begin{solution}
			
		\end{solution}
		
		\question\label{ques:IR:2.5} Use the result of Exercise 4 to show that there are infinitely many primes. (This proof is due to G. Polya.)
		
		\begin{solution}
			
		\end{solution}
		
		\question\label{ques:IR:2.6} For a rational number $ r $ let $ [r] $ be the largest integer less than or equal to $ r $, e.g., $ [\sfrac{1}{2}] = O $, $ [2] = 2 $, and $ [3\sfrac{1}{3}] = 3 $. Prove $ \mathrm{ord}_p n! = [n/p] + [n/p^{2}] + [n/p^{3}] + \cdots $.
		
		\begin{solution}
			
		\end{solution}
		
		\question\label{ques:IR:2.9} A function on the integers is said to be multiplicative if $ f(ab) = f(a)f(b) $ whenever $ (a, b) = 1 $. Show that a multiplicative function is completely determined by its value on prime powers.
		
		\begin{solution}
			
		\end{solution}
		
		\question\label{ques:IR:2.10} If $ f(n) $ is a multiplicative function, show that the function $ g(n) = \sum_{d|n} f(d) $ is also multiplicative.
		
		\begin{solution}
			
		\end{solution}
		
		\question\label{ques:IR:2.11} Show that $ \phi(n)=n\sum_{d|n} \mu(d)/d $ by first proving that $ \mu(d)/d $ is multiplicative and then using Exercises 9 and 10.
		
		\begin{solution}
			
		\end{solution}
		
		\question\label{ques:IR:2.15} Show that
		\begin{parts}
			\part\label{part:IR:2.15a} $ \sum_{d|n} \mu(n/d)\nu(d)=1 $ for all $ n $.
			
			\begin{solution}
				
			\end{solution}
			
			\part\label{part:IR:2.15b} $ \sum_{d|n}\mu(n/d)\sigma(d) = n $ for all $ n $.
			
			\begin{solution}
				
			\end{solution}
			
		\end{parts}
		
		\question\label{ques:IR:2.26} Verify the formal identities
		\begin{parts}
			\part\label{part:IR:2.26a} $ \zeta(s)^{-1}=\sum_{n=1}^{\infty}\mu(n)/n^{s} $.
			
			\begin{solution}
				
			\end{solution}
			
			\part\label{part:IR:2.27b} $ \zeta(s)^{2}=\sum_{n=1}^{\infty}\nu(n)/n^{s} $.
			
			\begin{solution}
				
			\end{solution}
			
			\part\label{part:IR:2.27c} $ \zeta(s)\zeta(s-1)=\sum_{n=1}^{\infty}\sigma(n)/n^{s} $.
			
			\begin{solution}
				
			\end{solution}
			
		\end{parts}
		
		\question\label{ques:IR:2.27} Show that $ \sum'1/n $, the sum being over square free integers, diverges. Conclude that $ \prod_{p<N}(l + 1/p)\to\infty $ as $ N \to \infty $. Since $ \mathrm{e}^{x}>1+x $, conclude that $ \sum_{p<N}1/p\to\infty $. (This proof is due to I. Niven.)
		
		\begin{solution}
			
		\end{solution}
		
		\question\label{ques:BMST:1.1} Mostre que a fração $ \dfrac{21n+4}{14n+3} $ é irredutível para todo $ n $ natural.
		
		\begin{solution}
			
		\end{solution}
		
		\question\label{ques:BMST:1.3} Demonstre:
		\begin{parts}
			\part\label{part:BMST:1.3a} se $ m|a-b $, então $ m|a^{k}-b^{k} $ para todo natural $ k $.
			
			\begin{solution}
				
			\end{solution}
			
			\part\label{part:BMST:1.3b} se $ f(x) $ é um polinômio com coeficientes inteiros e $ a $ e $ b $ são inteiros quaisquer, então $ a-b|f(a)-f(b) $.
			
			\begin{solution}
				
			\end{solution}
			
			\part\label{part:BMST:1.3c} se $ k $ é um natural ímpar, então $ a+b|a^{k}+b^{k} $.
			
			\begin{solution}
				
			\end{solution}
			
		\end{parts}
	
		\question\label{ques:BMST:1.5} Demonstrar que $ (n-1)^{2}|n^{k}-1 $ se, e só se, $ n-1|k $.
		
		\begin{solution}
			
		\end{solution}
		
		\question\label{ques:BMST:1.12} Seja $ F_n $ o n-ésimo termo da sequência de Fibonacci.
		
		\begin{parts}
			\part\label{part:BMST:1.12a} Encontrar dois números inteiros $ a $ e $ b $ tais que $ 233a+144b=1 $ (observe que $ 233 $ e $ 144 $ são termos consecutivos da sequência de Fibonacci).
			
			\begin{solution}
				
			\end{solution}
			
			\part\label{part:BMST:1.12b} Mostre que $ \mathrm{mdc}(F_n,F_{n+1})=1 $ para todo $ n\geqslant0 $.
			
			\begin{solution}
				
			\end{solution}
			
			\part\label{part:BMST:1.12c} Determine $ x_n $ e $ y_n $ tais que $ F_n\cdot x_n+F_{n+1}\cdot y_n=1 $.
			
			\begin{solution}
				
			\end{solution}
			
		\end{parts}
	
		\question\label{ques:BMST:1.19} Demonstrar que $ \mathrm{mdc}(2^{a}-1,2^{b}-1)=2^{\mathrm{mdc}(a,b)}-1 $ para todo $ a,b\in\mathbb{N} $.
		
		\begin{solution}
			
		\end{solution}
		
		\question\label{ques:BMST:1.20} Encontrar todas as funções $ f:\mathbb{Z}\times \mathbb{Z}\longrightarrow \mathbb{Z} $ satisfazendo simultaneamente as seguintes propriedades
		\begin{enumerate}[label=(\roman*)]
			\item\label{part:BMST:1.20a} $ f(a,a)=a $.
			\item\label{part:BMST:1.20b} $ f(a,b)=f(b,a) $.
			\item\label{part:BMST:1.20c} Se $ a>b $, então $ f(a,b)=\dfrac{a}{a-b}f(a-b,b) $.
		\end{enumerate}
		
		\begin{solution}
			
		\end{solution}
	
		\question\label{ques:BMST:1.21} Mostre que se $ n $ é um número natural composto, então $ n $ é divísivel por um primo $ p $ com $ p\leqslant \lfloor \sqrt{n}\rfloor $.
		
		\begin{solution}
			
		\end{solution}
		
	\end{questions}
\end{document}